%%%%%%%%%%%%%%%%%%%%%%%%%%%%%%%%%%%%%%%%%%%%%%%%%%%
% Signal Processing Laboratory (LTS5) - EPFL      %
% LaTeX student report template                   %
% Authors:                                        %
%   D. Perdios – dimitris.perdios@epfl.ch         %
%   A. Besson – adrien.besson@epfl.ch             %
% v0.1 - 22.12.16                                 %
% Typeset configuration: pdfLaTeX + Biber         %
%%%%%%%%%%%%%%%%%%%%%%%%%%%%%%%%%%%%%%%%%%%%%%%%%%%


% Known problems with biblatex-ieee style
% 1) still problems with arXiv (eprint) having a primaryClass (eprintclass) entry. The output is ok, i.e. arXiv: <eprint> [<primaryClass>], but the href is not correct and does not include the <primaryClass>. For exemple it would link to http://arxiv.org/abs/407151 instead of http://arxiv.org/abs/astro-ph/407151 (<arxiv>=40751, <primaryClass>=astro-ph).
%	Maybe a problem due to Mendeley export which should export <arxiv>=40751/astro-ph without any <primaryClass> field
% 2) if more than one url is given it seems not to be a list an adds %20 (space char) between each links...

%% LOAD OPTIONS
% 	Remark 1: backend, style, bibstyle and citestyle options as well natbib and mcite compatibility options must be set at loading time in the square brackets
%	Remark 2: \ExecuteBibliographyOptions[<entrytype, ...>]{<key=value, ...>} 
%	Remark 3: seems better than clearfield since it doesn't clear the field, just doesn't print
%\ExecuteBibliographyOptions{citestyle=numeric-comp} % nicer than the ieee citestyle
%\ExecuteBibliographyOptions{sorting=none} % sorting=none already defined by ieee
\ExecuteBibliographyOptions{mincitenames=3} % if more than maxcitename, truncates to mincitename
\ExecuteBibliographyOptions{maxcitenames=3} % if more than maxcitename, truncates to mincitename
\ExecuteBibliographyOptions{minbibnames=6} % if more than maxbibname, truncates to minbibname
\ExecuteBibliographyOptions{maxbibnames=6} % if more than maxbibname, truncates to minbibname
%\ExecuteBibliographyOptions{isbn=false}
%\ExecuteBibliographyOptions{doi=false}
%\ExecuteBibliographyOptions{eprint=false}
%\ExecuteBibliographyOptions{url=false} % does not affect @online since url is mandatory 
%\ExecuteBibliographyOptions{firstinits=true} % already defined by ieee bibstyle
%\ExecuteBibliographyOptions{date=year}

%% SOME BASIC CUSTOMIZATIONs
% 	Bilbiography font size
%\renewcommand*{\bibfont}{\small}

% 	Bibliography item separation
%\setlength\bibitemsep{0pt}		% No separation between bib entries

% 	Bibliography name
%TODO: change bibliography name in the .cls file by using \refname instead of \bibname for the chapter name

% At every bibitem
%TODO: how not to print an entry without clearing it
\AtEveryBibitem{% Clean up the bibtex rather than editing it
% 	\clearlist{address}
%	\clearfield{abstract}
%	\clearfield{title}
%	\clearname{author}
% 	\clearfield{date}
 	\clearfield{doi}
% 	\clearfield{eprint}
 	\clearfield{isbn}
 	\clearfield{issn}
 	\clearfield{isrn}
% 	\clearlist{location}
 	\clearfield{month}
 	\clearfield{number}
%	\clearfield{note}
%	\clearfield{url}
%	\clearfield{issue}
 	\clearfield{series}
% 	\clearname{editor}
 	
 	% Remove arXiv (eprint) for @article if journaltitle (which is converted from .bib journal entry) exists
 	\ifentrytype{article}
 		{
 			\iffieldundef{journaltitle}
 				{}
 				{\clearfield{eprint}}
 		}
 		{}
 	% Remove arXiv (eprint) for @inproceedings if booktitle exists
	\ifentrytype{inproceedings}
 		{
 			\iffieldundef{booktitle}
 				{}
 				{\clearfield{eprint}}
 		}
 		{}
	% Remove publisher, location and editor except for @book
 	\ifentrytype{book}
 		{}
 		{
  			\clearlist{publisher}
  			\clearlist{location}
  			\clearname{editor}
		}
	% Remove url except for @online (similar to using url=false package option)
	\ifentrytype{online}
 		{}
 		{\clearfield{url}}
}

% At every citekey
%TODO: check \AtEveryCitekey
%\AtEveryCitekey{
%	\clearfield{month}
%}

% Preserve acronyms in titles which are lowcased by ieee style
%	Remark 1: the expression \b\w*[A-Z]{2,}\w*\b finds a word containing at least 2 capital letters
%	Remark 2: (expr) group elements of the expression and capture tokens
%	Remark 3: replace expr wrapping it in curly braces (don't know why the empty group is needed, see http://tex.stackexchange.com/questions/238078/biblatex-preserve-case-of-acronyms-in-title)
%	Remark 4: changing [A-Z]{2,} to [A-Z]{1,} would preserve the uppercase of every words
%TODO: check with 2016 version of TeX and new versions of biblatex-ieee yet freezed
\DeclareSourcemap{
	\maps[datatype=bibtex]{
		\map{
			\step[fieldsource=title, match=\regexp{(\b\w*[A-Z]{2,}\w*\b)}, replace={{}{$1}}]
		}
	}
}

% More intelligent initials, for exemple, Ph. for Philippe rather than P. (see: http://tex.stackexchange.com/questions/295476/two-or-three-letter-initials-in-bibliography-with-biblatex)
\DeclareStyleSourcemap{%
	\maps[datatype=bibtex]{%
	\map{%
		% Author field
		\step[fieldsource=author,%
			match={\regexp{\b(Chr|Ch|Th|Ph|[B-DF-HJ-NP-TV-XZ](l|r))(\S*,)}},%
			replace={\regexp{\{$1\}$3}}]% Protect last names (first last)
		\step[fieldsource=author,%
			match={\regexp{([^,]\s)\b(Chr|Ch|Th|Ph|[B-DF-HJ-NP-TV-XZ](l|r))}},%
			replace={\regexp{$1\{$2\}}}]% Protect last names (last, first)
		\step[fieldsource=author,%
			match={\regexp{\b(Chr|Ch|Th|Ph|[B-DF-HJ-NP-TV-XZ](l|r))([^\}])}},%
			replace={\regexp{\{\\relax\{\}$1\}$3}}]% Insert \relax after abbreviating
		% Editor field
		\step[fieldsource=editor,%
			match={\regexp{\b(Chr|Ch|Th|Ph|[B-DF-HJ-NP-TV-XZ](l|r))(\S*,)}},%
			replace={\regexp{\{$1\}$3}}]% Protect last names (first last)
		\step[fieldsource=editor,%
			match={\regexp{([^,]\s)\b(Chr|Ch|Th|Ph|[B-DF-HJ-NP-TV-XZ](l|r))}},%
			replace={\regexp{$1\{$2\}}}]% Protect last names (last, first)
		\step[fieldsource=editor,%
			match={\regexp{\b(Chr|Ch|Th|Ph|[B-DF-HJ-NP-TV-XZ](l|r))([^\}])}},%
			replace={\regexp{\{\\relax\{\}$1\}$3}}]% Insert \relax after abbreviating
}}}%


% Typeset only the first page with p. instead of pp. for any entry
%\DeclareFieldFormat{pages}{\mkfirstpage[{\mkpageprefix[bookpagination]}]{#1}}

% Make volume typset bold
%\DeclareFieldFormat[article,periodical]{volume}{\mkbibbold{#1}}

% Removing the in
% 		for every entries
%\renewbibmacro{in:}{}
%% 		only for articles
%\renewbibmacro{in:}{%
%  \ifentrytype{article}{}{\printtext{\bibstring{in}\intitlepunct}}}