\chapter{Exploratory Data Analysis}
\label{chapter_data_exploration}

As a preliminary step, we propose to perform an exploratory data analysis. This will give us some hints about the dataset, \eg{} the most important features, their correlation etc.

In order to have a first overview of the features, summary statistics are reported in Table~\ref{tab_data_exploration}. It can be noticed that the frequencies have low values, which makes sense since they are expressed in \si{\kilo\hertz}. 
The mean fundamental frequency is about \SI{143}{\hertz} which is coherent with the fundamental frequencies of males and females reported in the literature~\cite{Traunmller1994}. 

Regarding the shape of the spectrum, the mean value of ``skew'' indicates an average right-asymmetry of the spectrum. The mean value of ``kurt'' shows that the frequency distribution is leptokurtic. About the flatness of the spectrum, the mean values of ``sfm'' and ``sp.ent'' seem to have an inconsistent behaviour since one is above \num{0.5} and the other is below. However, the high standard deviation of ``sfm'' makes the analysis rather difficult.

%%%%%%%%%%%% TABLE %%%%%%%%%%%
\begin{table}[htb]
	\ra{1.2}
	\caption{Summary statistics of the features of the dataset}
	 \begin{subtable}{\textwidth}
			\centering
			\begin{tabular}{@{} c c c c  c c c c c c @{}}\toprule
				& meanfreq & sd & median & Q25 & Q75 & skew & kurt & sp.ent & sfm \\
			mean & \num{0.181} &\num{0.0571} & \num{0.186} & \num{0.140} & \num{0.225} & \num{3.14} &\num{36.6} & \num{0.895} &	\num{0.408} \\
			std & \num{0.0299} & \num{0.0167} & \num{0.0364} & \num{0.0487} & \num{0.0236} & \num{4.24} & \num{134} & \num{0.0450} & \num{0.178} \\
			min & \num{0.0394} & \num{0.0184} & \num{0.0110} & \num{0.000229} & \num{0.0429} & \num{0.142} & \num{2.07} & \num{0.739} & \num{0.0369} \\
			\SI{25}{\percent} & \num{0.164} & \num{0.0420} & \num{0.170} & \num{0.111} & \num{0.209} & \num{1.65} & \num{5.67} & \num{0.863} & \num{0.258} \\
			\SI{50}{\percent} & \num{0.185} & \num{0.0592} & \num{0.190} & \num{0.140} & \num{0.226} & \num{2.20} & \num{8.32} & \num{0.902} & \num{0.396} \\
			\SI{75}{\percent} & \num{0.199} & \num{0.0670} & \num{0.211}  & \num{0.176} & \num{0.244} & \num{2.93} & \num{13.7} & \num{0.929} & \num{0.534} \\
			max & \num{0.251} & \num{0.115} & \num{0.261} & \num{0.247} &
			\num{0.273} & \num{34.7} & \num{1310} & \num{0.982} & \num{0.843} \\ \bottomrule
			\end{tabular}
		\end{subtable}\hfill\null%

		\begin{subtable}{\textwidth}
			\centering
			\begin{tabular}{@{} c c c c  c c c c c @{}}\toprule
			 & mode & meanfun & minfun & maxfun & meandom & mindom & maxdom & modindx \\
			mean & \num{0.165} & \num{0.143} & \num{0.0368} & \num{0.259} & \num{0.829} & \num{0.0526} & \num{5.05} & \num{0.174} \\
			std & \num{0.0772} & \num{0.0323} & \num{0.0192} & \num{0.0301} & \num{0.525} & \num{0.0633} & \num{3.52} & \num{0.119} \\
			min & \num{0.00} & \num{0.0556} & \num{0.00977} & \num{0.103} & \num{0.00781} & \num{0.00488} & \num{0.00781} & \num{0.00} \\
			\SI{25}{\percent} & \num{0.118} & \num{0.117} & \num{0.0182} & \num{0.254} & \num{0.420} & \num{0.00781} & \num{2.07} & \num{0.0998} \\
			\SI{50}{\percent} & \num{0.187} & \num{0.140} & \num{0.0461} & \num{0.271} & \num{0.766} & \num{0.0234} & \num{4.99} & \num{0.139} \\
			\SI{75}{\percent} & \num{0.221} & \num{0.170} & \num{0.0479} & \num{0.277} & \num{1.18} & \num{0.0703} & \num{7.01} & \num{0.210} \\
			max & \num{0.280} & \num{0.238} & \num{0.204} & \num{0.279} & \num{2.96} & \num{0.459} & \num{21.87} & \num{0.932} \\ \bottomrule	
			\end{tabular}
		\end{subtable}\hfill\null%
	\label{tab_data_exploration}
\end{table}
%%%%%%%%%%%% END OF TABLE %%%%%%%%%%%

Regarding the distribution of the samples, there are \num{1584} recording of male voices and \num{1584} recording of female voices. So the classes are perfectly balanced.

Let us have a look to the correlation between the features. The correlation matrix, displayed in Figure~\ref{fig_corr_matrix}, exhibits high correlations between ``skew'' and ``kurt'' and between ``sp.ent'' and ``sfm'', which make sense since they quantify similar quantities. It can also be noticed that ``meanfreq'', ``median'', ``Q25'', ``Q75'' are highly correlated which is self-evident given their definitions. 
%%%%%%%%%%%% FIGURE %%%%%%%%%%%
\newlength{\BoxPlotFigWidth}
\newlength{\BoxPlotFigHeight}
\setlength{\BoxPlotFigWidth}{0.8\textwidth}
\settoheight{\BoxPlotFigHeight}{\includegraphics[width=\BoxPlotFigWidth]{figures/correlation_matrix.pdf}}
\begin{figure}[htb]
	\centering
	\includegraphics[height=\BoxPlotFigHeight]{figures/correlation_matrix.pdf}
	\caption{Correlation matrix of the dataset.}
	\label{fig_corr_matrix}
\end{figure}
%%%%%%%%%%%% END OF FIGURE %%%%%%%%%%%
%%%%%%%%%%%% FIGURE %%%%%%%%%%%
\setlength{\BoxPlotFigWidth}{0.48\textwidth}
\settoheight{\BoxPlotFigHeight}{\includegraphics[width=\BoxPlotFigWidth]{figures/meanfreq_boxplot.pdf}}
\begin{figure}[htb]
	% Maximum length
	\hfill%
	\subcaptionbox{\label{fig_meanfreq_box}}{\includegraphics[height=\BoxPlotFigHeight]{figures/meanfreq_boxplot.pdf}}\hfill%
	\subcaptionbox{\label{fig_meanfun_box}}{\includegraphics[height=\BoxPlotFigHeight]{figures/meanfun_boxplot.pdf}}\hfill\null%
	\caption{Box plots for (a)-``meanfreq'' and (b)-``meanfun'' features.}
	\label{fig_mean_freq}
\end{figure}
%%%%%%%%%%%% END OF FIGURE %%%%%%%%%%%

In the state-of-the-art, it appears that the fundamental frequency is a key feature for AGR, as stated in Chapter~\ref{chap:introduction}. Intuitively, we also think that the mean frequency should be a good classifier. In order to analyze this, Figs.~\ref{fig_meanfreq_box} and~\ref{fig_meanfun_box} represent the box plots of ``meanfreq'' and ``meanfun'' respectively. 
It can be noticed that ``meanfun'' is indeed a key feature for classification since the overlap between male and female distributions is very low. Regarding ``meanfreq'', the overlap is bigger than for ``meanfun''. 

Figs.~\ref{fig_meanfreq_facet} and~\ref{fig_meanfun_facet} represent the male and female distributions with respect to ``meanfreq'' and ``meanfun'' respectively. They substantiate the analysis made with the box plot, \ie{} that ``meanfun'' is a key component in AGR and is a far better classifier than ``meanfreq''.
%%%%%%%%%%%% FIGURE %%%%%%%%%%%
\settoheight{\BoxPlotFigHeight}{\includegraphics[width=\BoxPlotFigWidth]{figures/meanfreq_facetgrid.pdf}}
\begin{figure}[htb]
	% Maximum length
	\hfill%
	\subcaptionbox{\label{fig_meanfreq_facet}}{\includegraphics[height=\BoxPlotFigHeight]{figures/meanfreq_facetgrid.pdf}}\hfill%
	\subcaptionbox{\label{fig_meanfun_facet}}{\includegraphics[height=\BoxPlotFigHeight]{figures/meanfun_facetgrid.pdf}}\hfill\null%
	\caption{Distribution of ``male'' and ``female'' for (a)-``meanfreq'' and (b)-``meanfun'' features.}
	\label{fig_facet}
\end{figure}
%%%%%%%%%%%% END OF FIGURE %%%%%%%%%%%

