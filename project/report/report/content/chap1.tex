%%%%%%%%%%%%%%%%%%%%%%%%%%%%%%%%%%%%%%%%%%%%%%%%%%%
% Signal Processing Laboratory (LTS5) - EPFL      %
% LaTeX student report template                   %
% Authors:                                        %
%   D. Perdios – dimitris.perdios@epfl.ch         %
%   A. Besson – adrien.besson@epfl.ch             %
% v0.1 - 22.12.16                                 %
% Typeset configuration: pdfLaTeX + Biber         %
%%%%%%%%%%%%%%%%%%%%%%%%%%%%%%%%%%%%%%%%%%%%%%%%%%%


\chapter{Introduction}
\label{chap:introduction}

In the last decades, automatic gender recognition~(AGD) from speech has grown many interest thanks to the digitization of an extensive number of applications and the development of mobile platforms~\cite{Wu_JASA_1991, Wu_JASA_1991_2, Childers_ICASSP_1988, Harb2005, Zeng2006, Sorokin2008, Metze_ICASSP_2007, Bocklet_ICASSP_2008}.

The applications of AGR have increased consequently. Indeed, in general, the accuracy of gender-dependent systems is higher than the one of gender-independent systems~\cite{Harb2005}. Thus, AGR improves the prediction of other speaker traits such as age~\cite{Levi2015} and emotional state~\cite{Bisio2013, Ververidis2004}. It can also facilitate speech recognition by gender-based normalization~\cite{Wegmann_ICASSP_1996} and is a key feature for more natural and personalized dialog systems such as Siri.

The AGR techniques are based on statistical features extracted from the speech signals such as maximum, minimum and average frequency measured in a time span. These features translate physiological differences between males and females like the length of the vocal chords or the glottal shape~\cite{Titze_JASA_1989}. Among all the features, it appears that the fundamental frequency plays a crucial role in gender classification as described in many studies~\cite{Hollien1967, Wu_JASA_1991, Poon2015}. In recent works, the use of the fundamental frequency coupled with spectral components such as Mel-frequency spectral components~\cite{Gupta2016} or relative spectral perceptual linear predictive coefficients~\cite{Zeng2006} have demonstrated best AGR performances even in noisy environments.

In this project, we study different state-of-the-art classification methods applied to the task of gender recognition by voice. The study is based on a dataset of features extracted from \num{3168} subjects available on Kaggle\footnote{\url{https://www.kaggle.com/primaryobjects/voicegender}} and described in details in Chapter~\ref{chap:dataset}. A preliminary exploratory data analysis is performed in Chapter~\ref{chapter_data_exploration} which leads us to a first intuitive classification technique described in Chapter~\ref{chap_intuitive_approach}. Starting from the conclusions of this intuitive approach, the exhaustive comparison of the methods is achieved in Chapter~\ref{chapter_classification} and the best model is selected. Eventually, the best model is tested on $\num{4}$ voices recorded by the authors in Chapter~\ref{chap_test_our_dataset}.
  