\chapter{Application of the Best Classifier on Acquired Voice Recordings}
\label{chap_test_our_dataset}

One main disadvantage of the dataset provided for the study is that there is no information about the way the voices have been recorded. This prevents us from further study and characterization of the classification such as its robustness against noise.

To study this aspect, we propose to test the best classifier on a small dataset of voice recordings that we have acquired. PUT CHARACTERISTICS OF THE DATASET
Once acquired, a ``.csv" file containing the features is generated using available R script~\footnote{https://github.com/primaryobjects/voice-gender/blob/master/sound.R}. 

First, XGBoost, which has been identified as the best classifier in Section~\ref{sec_our_strat} is fitted on the whole dataset. Then it is used to classify the acquired voices, based on the features extracted from the ``.csv" file.

